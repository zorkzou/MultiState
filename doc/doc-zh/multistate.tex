\documentclass[UTF8]{ctexart}
\usepackage{graphicx}
\usepackage{float}
\usepackage{mhchem}
\usepackage{amsmath}
\usepackage{geometry}
\geometry{a4paper,centering,scale=0.8}
\usepackage[format=hang,font=small,textfont=it]{caption}
\usepackage[nottoc]{tocbibind}

\usepackage{listings,xcolor}
\lstset{
%        extendedchars=false,          % non-English characters
        lineskip=3pt,
        basicstyle=\tt\small\color{black},
        commentstyle=\tt\color{red!60!black},
        morecomment=[l]\!,
        keywordstyle=\tt\color{green!70!black},
        stringstyle=\tt\color{green},
        showspaces=false,             % underline spaces in the codes
        showstringspaces=false,       % underline spaces only in a string
        showtabs=false,               % underline tabs in the codes
        numberstyle=\tiny\color{black},
        numbersep=14pt,               % how far the line-numbers are from the code
        texcl=true,                   % comments in LaTeX if true
        emph={                        % keywords to be highlighted
%               subroutine, return,
%               end
        },
        emphstyle=\sf\bfseries\color{red!50!black}\fcolorbox{orange!40!white}{.},
        numbers=left,
        rulecolor=\color{blue},
        frame=single
%        tabsize=2,                    % set default tab-size to 2 spaces
%        frame=shadowbox, rulesepcolor=\color{blue}
}

\newenvironment{myquote}
{\begin{quote}\kaishu\zihao{-5}}
{\end{quote}}

\newcommand\degree{^\circ}

\title{\heiti 自旋禁戒反应的近似旋轨耦合处理程序MS@GWEV}

\date{\today}

\begin{document}

\maketitle

\vspace{10mm}
\tableofcontents

\newpage

\section{理论}

\subsection{Truhlar的两态混合SO模型哈密顿及能量导数}

旋轨耦合效应在微扰级别(也即Russell-Saunders耦合模型或LS耦合模型\cite{ref1})可分为一阶旋轨耦合和二阶旋轨耦合,其中前者来自单个自旋轨道态(也称LS态)因自旋多重度导致的自身分裂,例如三重态分裂为三项,五重态分裂为五项;后者来自不同LS态之间的旋轨耦合相互作用。
一阶旋轨耦合为主的旋量态更多地保留了原LS态的性质,标量方法已经定性正确。只有二阶旋轨耦合起作用的化学反应,才会出现不一样的性质,这是人们更关心的。

若忽略一阶旋轨耦合,由两个LS态(假设自旋多重度分别为$m$、$n$)构成的$m+n$阶自旋轨道矩阵可以简化为以下的2阶有效模型哈密顿矩阵\cite{ref2,ref3}
\begin{align}\label{tsr-eq01}
\mathbf{H} =
\left[\begin{array}{cc}
    V^m(X) &  \chi \\
    \chi & V^n(X)
\end{array}\right],
\end{align}
其中,$V^m(X)$,$V^n(X)$是两个自旋态在各自势能面上$X$点位置的能量(以下略去坐标变量$X$),$\chi$是自旋轨道耦合系数,近似可看作是不依赖于$X$的经验参数。Takayanagi和Nakatomi的测试表明,对于铁化合物$\chi$取50--400 cm$^{-1}$以内的任意值对结果影响不大\cite{ref2},但Yang等建议取231.6 meV(1868 cm$^{-1}$)。对于钨化合物的建议值为302 meV(2436 cm$^{-1}$)\cite{ref3}。

\eqref{tsr-eq01}式可以展开成一元二次方程,有两个根$e_1$、$e_2$,对应两个自旋混合态的能量。我们关心较低一个根的能量$e_1$
\begin{align}\label{tsr-eq02}
e_1 = \frac{1}{2}\left(V^m + V^n - \beta\right)  \, \rm .
\end{align}

由公式\eqref{tsr-eq02}可以很容易推导出$e \equiv e1$关于坐标$X_\mu$的梯度和$X_\mu$、$X_\nu$的Hessian\cite{ref2,ref3}:
\begin{align}\label{tsr-eq03}
e_\mu & \equiv \frac{\partial e}{\partial X_\mu} = \frac{1-\alpha}{2}V^m_\mu + \frac{1+\alpha}{2}V^n_\mu  \, \rm , \\ \label{tsr-eq04}
e_{\mu,\nu} & \equiv \frac{\partial^2 e}{\partial X_\mu\partial X_\nu} = \frac{1-\alpha}{2}V^m_{\mu,\nu} + \frac{1+\alpha}{2}V^n_{\mu,\nu} + \frac{\alpha^2-1}{2\beta}\left(V^m_\mu-V^n_\mu\right)\left(V^m_\nu-V^n_\nu\right)  \, \rm .
\end{align}

在以上公式中,
\begin{align}\label{para-eq01}
\alpha &= \frac{V^m - V^n}{\beta} \, \rm , \\ \label{para-eq02}
\beta &= \sqrt{(V^m - V^n)^2 + 4\chi^2} \, \rm .
\end{align}

\subsection{多态混合SO模型哈密顿及能量导数}

如果反应路径涉及两个以上的自旋轨道态,用以上方法可以分段、分组做两态计算,然后拼接在一起,取能量最低路径\cite{ref4}。这种处理并不方便,并且如果两个以上的自旋态在某点附近发生较强的耦合,则以上方法失效。
为此,我们对一般的有效模型哈密顿矩阵推导出一阶和二阶解析导数。它适用于对于两个、三个或更多自旋态混合的情况。

以三种自旋态的混合为例,SO模型哈密顿为:
\begin{align}\label{hso0-eq01}
\mathbf{H} = \left[\begin{array}{ccc}
    V_1 & \chi_{1,2} & \chi_{1,3} \\
    \chi_{1,2} & V_2 & \chi_{2,3} \\
    \chi_{1,3} & \chi_{2,3} & V_3
\end{array}\right]
\end{align}
其中$V_i$是第$i$个自旋态在其势能面上$X$点位置的能量,$\chi_{i,j}$是$i$、$j$两态之间的经验旋轨耦合常数($i,j$=1,2,3)。
为了简单,\eqref{hso0-eq01}式做了如下简化:
\begin{enumerate}
\item 考虑到两态计算中,计算结果对$\chi$值不是很敏感,因此任意两态之间的$\chi$采用相同的值
\item 为了确保$\mathbf{Q}_{(\mu)}$矩阵的第一列非奇异(见\eqref{app1-eq05}式),需要避免偶然简并的自旋混合基态,$\mathbf{H}$的非对角元要取负值(对比:在两态计算中,非对角元符号不影响结果,且两个特征值不相等)
\end{enumerate}

简化后的\eqref{hso0-eq01}式变为\eqref{hso0-eq02}式:
\begin{align}\label{hso0-eq02}
\mathbf{H} = \left[\begin{array}{ccc}
    V_1 & -\chi & -\chi \\
    -\chi & V_2 & -\chi \\
    -\chi & -\chi & V_3
\end{array}\right]
\end{align}
其中,$\chi$ > 0。与两态计算相同,在以下推导中忽略$\chi$对坐标的依赖。

$\mathbf{H}$的特征值方程为:
\begin{align}\label{hso0-eq03}
\mathbf{H} \, \mathbf{D} = \mathbf{D} \, \mathbf{E}
\end{align}
其中特征值矩阵$\mathbf{E}$的对角元包含三个特征值$e_1 < e_2 \leq e_3$,特征矢量矩阵$\mathbf{D}$包含三个列矢量。

\subsubsection{SO模型哈密顿的一阶导数}

$\mathbf{H}$对坐标$X_\mu$求导,得:
\begin{align}\label{hso1-eq01}
\mathbf{H}_\mu \, \mathbf{D} + \mathbf{H} \, \mathbf{D}_\mu = \mathbf{D}_\mu \, \mathbf{E} +  \mathbf{D} \, \mathbf{E}_\mu
\end{align}
$\mathbf{D}^\dagger$从左侧乘以\eqref{hso1-eq01}式:
\begin{align}\label{hso1-eq02}
\mathbf{D}^\dagger \, \mathbf{H}_\mu \, \mathbf{D} + \mathbf{E} \, \mathbf{D}^\dagger \, \mathbf{D}_\mu = \mathbf{D}^\dagger \, \mathbf{D}_\mu \, \mathbf{E} +  \mathbf{E}_\mu
\end{align}

令$\mathbf{D}_\mu = \mathbf{D} \, \mathbf{Q}_{(\mu)}$,代入\eqref{hso0-eq02}式后得到:
\begin{align}\label{hso1-eq03}
\mathbf{E}_\mu = \mathbf{D}^\dagger \, \mathbf{H}_\mu \, \mathbf{D} + \mathbf{E} \, \mathbf{Q}_{(\mu)} - \mathbf{Q}_{(\mu)} \, \mathbf{E}
\end{align}
上式等号左侧为对角矩阵;由于$\mathbf{Q}_{(\mu)}$是对角元为0的反对称矩阵(见后),等号右侧后两项的对角元为0,对等号左侧对角元无贡献,因此\eqref{hso1-eq03}式可以简化为
\begin{align}\label{hso1-eq04}
\mathbf{E}_\mu = \left[\mathbf{D}^\dagger \, \mathbf{H}_\mu \, \mathbf{D}\right]_{\rm diag}
\end{align}
其中下标diag表示取对角项。对于我们关心的自旋混合基态(能量$e_1$简写为$e$),梯度为
\begin{align}\label{hso1-eq05}
e_\mu = \mathbf{d}^\dagger \, \mathbf{H}_\mu \, \mathbf{d}
\end{align}
其中$\mathbf{d}$是$\mathbf{D}$的第一个列矢量。由于$\chi$不依赖坐标,上式中的$\mathbf{H}_\mu$是对角矩阵,由三个自旋态梯度矢量的第$\mu$个分量构成。

事实上,\eqref{hso1-eq05}式也可通过Hellmann-Feynman定理直接得到。

\subsubsection{SO模型哈密顿的二阶导数}

根据\eqref{hso1-eq01}式,自旋混合基态的梯度需满足以下方程:
\begin{align}\label{hso2-eq01}
\mathbf{H}_\mu \, \mathbf{d} + \mathbf{H} \, \mathbf{d}_\mu = e \, \mathbf{d}_\mu + e_\mu \, \mathbf{d}
\end{align}
上式对坐标$X_\nu$求导,得:
\begin{align}\label{hso2-eq02}
\mathbf{H}_{\mu\nu} \, \mathbf{d} + \mathbf{H}_{\mu} \, \mathbf{d}_{\nu} +
\mathbf{H}_{\nu} \, \mathbf{d}_\mu + \mathbf{H} \, \mathbf{d}_{\mu\nu}
= e \, \mathbf{d}_{\mu\nu} + e_{\nu} \, \mathbf{d}_\mu +
 e_\mu \, \mathbf{d}_{\nu} + e_{\mu\nu} \, \mathbf{d}
\end{align}

$\mathbf{d}^\dagger$从左侧乘以\eqref{hso2-eq02}式:
\begin{align}\label{hso2-eq03}
\mathbf{d}^\dagger \, \mathbf{H}_{\mu\nu} \, \mathbf{d} + \mathbf{d}^\dagger \, \mathbf{H}_{\mu} \, \mathbf{d}_{\nu} +
\mathbf{d}^\dagger \, \mathbf{H}_{\nu} \, \mathbf{d}_\mu + \mathbf{d}^\dagger \, \mathbf{H} \, \mathbf{d}_{\mu\nu}
= e \, \mathbf{d}^\dagger \, \mathbf{d}_{\mu\nu} + e_{\nu} \, \mathbf{d}^\dagger \, \mathbf{d}_\mu +
e_\mu \, \mathbf{d}^\dagger \, \mathbf{d}_{\nu} + e_{\mu\nu}
\end{align}

在\eqref{hso2-eq03}式中:等号左侧第四项和等号右侧第一项相等(利用\eqref{hso0-eq03}式),可以消去;由于
$\mathbf{D}^\dagger \, \mathbf{D}_\mu = \mathbf{D}^\dagger \, \mathbf{D} \, \mathbf{Q}_{(\mu)} = \mathbf{Q}_{(\mu)}$,且$\mathbf{Q}_{(\mu)}$矩阵的对角元为0,可知$\mathbf{d}^\dagger \, \mathbf{d}_\mu = 0$。于是\eqref{hso2-eq03}式可化简为:
\begin{align}\label{hso2-eq04}
e_{\mu\nu} &= \mathbf{d}^\dagger \, \mathbf{H}_{\mu\nu} \, \mathbf{d} + \mathbf{d}^\dagger \, \mathbf{H}_{\mu} \, \mathbf{d}_{\nu} +
\mathbf{d}^\dagger \, \mathbf{H}_{\nu} \, \mathbf{d}_\mu \\ \label{hso2-eq05}
&= \mathbf{d}^\dagger \, \mathbf{H}_{\mu\nu} \, \mathbf{d} + \mathbf{d}^\dagger \, \mathbf{H}_{\mu} \, \mathbf{D} \, \mathbf{q}_{(\nu)} +
\mathbf{d}^\dagger \, \mathbf{H}_{\nu} \, \mathbf{D} \, \mathbf{q}_{(\mu)}
\end{align}
其中,$\mathbf{q}_{(\mu)}$和$\mathbf{q}_{(\nu)}$分别是$\mathbf{Q}_{(\mu)}$和$\mathbf{Q}_{(\nu)}$的第一列,需要额外计算(见下一节)。
借助\eqref{app1-eq06}式,可知\eqref{hso2-eq05}式等号右侧的第二项和第三项是相等的,因此\eqref{hso2-eq04}式可以简化为
\begin{align}\label{hso2-eq06}
e_{\mu\nu} = \mathbf{d}^\dagger \, \mathbf{H}_{\mu\nu} \, \mathbf{d} + 2 \, \mathbf{d}^\dagger \, \mathbf{H}_{\mu} \, \mathbf{D} \, \mathbf{q}_{(\nu)}
\end{align}

由于$\chi$不依赖坐标,上式中的$\mathbf{H}_{\mu\nu}$、$\mathbf{H}_\mu$、$\mathbf{H}_\nu$都是3$\times$3对角矩阵。


\subsubsection{$\mathbf{Q}_{(\mu)}$矩阵的计算}

根据
\begin{align}\label{app1-eq01}
\mathbf{D}^\dagger \, \mathbf{D} = \mathbf{I}
\end{align}
其中$\mathbf{I}$是单位矩阵,\eqref{app1-eq01}式对坐标$X_\mu$求导得:
\begin{align}\label{app1-eq02}
\mathbf{D}_\mu^\dagger \, \mathbf{D} = -\mathbf{D}^\dagger \, \mathbf{D}_\mu
\end{align}
代入$\mathbf{D}_\mu = \mathbf{D} \, \mathbf{Q}_{(\mu)}$,得到:
\begin{align}\label{app1-eq03}
\mathbf{Q}_{(\mu)} = -\mathbf{Q}_{(\mu)}^\dagger
\end{align}

可见,$\mathbf{Q}_{(\mu)}$是反对称矩阵,对角元为0。非对角元可通过\eqref{hso1-eq03}式获得:
\begin{align}\label{app1-eq04}
\left[\mathbf{Q}_{(\mu)}\right]_{i, j\neq i} = \frac{\left[\mathbf{D}^\dagger \, \mathbf{H}_\mu \, \mathbf{D}\right]_{i, j}}{e_j - e_i}
\end{align}

在上一节的计算中,仅需要$\mathbf{Q}_{(\mu)}$的第一列(即$\mathbf{q}_{(\mu)}$),其表达式为:
\begin{align}\label{app1-eq05}
\left[\mathbf{q}_{(\mu)}\right]_{i\neq 1} = \frac{\left[\mathbf{D}^\dagger \, \mathbf{H}_\mu \, \mathbf{d}\right]_i}{e_1 - e_i}
\end{align}
或
\begin{align}\label{app1-eq06}
\mathbf{q}_{(\mu)} = \left(e_1 \, \mathbf{I} - \mathbf{E}\right)^{+1}\mathbf{D}^\dagger \, \mathbf{H}_\mu \, \mathbf{d}
\end{align}
其中“+1”表示伪逆矩阵。

\subsubsection{处理由更多自旋态构成的自旋混合态}

以上方法也适用于任何数量自旋态的多态反应计算(尽管除了单个过渡原子以外的四态以上体系很少报道),关键是如何构造SO模型哈密顿(\eqref{hso0-eq01}式),从而在Hessian计算中避免\eqref{app1-eq05}式的分母为0。一种方法是把$\mathbf{H}$的非对角元全部取负值(可用同一个值,也可用不同的多个值),可以保证自旋混合基态非简并;另一种方法是不限制$\mathbf{H}$非对角元的正负号,当\eqref{app1-eq05}式的分母接近0时,直接令$1/(e_1 - e_i) = 0$。
这是因为当两个能量简并时,存在2$\times$2转动矩阵$\mathbf{X}$使二者的特征矢量发生任意混合。由于$\mathbf{X}$存在无穷多可能性,但是却不影响最终结果,因此这部分的$\mathbf{q}_{(\mu)}$矩阵元只有取零值时才能满足要求。

\noindent
%\textbf{参考文献}
\begin{thebibliography}{00}
\bibitem{ref1} H. N. Russell and F. A. Saunders, \textit{New Regularities in the Spectra of the Alkaline Earths}, Astrophys. J. \textbf{61}, 38--69, (1925).
\bibitem{ref2} T. Takayanagi and T. Nakatomi, \textit{Automated Reaction Path Searches for Spin-Forbidden Reactions}, J. Comput. Chem. \textbf{39}, 1319--1326 (2018).
\bibitem{ref3} B. Yang, L. Gagliardi, and D. G. Truhlar, \textit{Transition States of Spin-Forbidden Reactions}, Phys. Chem. Chem. Phys. \textbf{20}, 4129--4136 (2018).
\bibitem{ref4} T. Takayanagi, \textit{Two-state reactivity in the acetylene cyclotrimerization reaction catalyzed by a single atomic transition-metal ion: The case for \ce{V+} and \ce{Fe+}},
Comput. Theor. Chem. \textbf{1211}, 113682 (2022).
\end{thebibliography}

\newpage

\section{程序和脚本}

MS@GWEV程序是\textsc{GWEV}套件的模块之一,为近似处理涉及多个自旋态的自旋禁戒反应而开发,用Fortran90语言编写。
运行方式为\textsc{Gaussian}程序通过MS@GWEV调用其它量子化学程序(包括\textsc{Gaussian}自身)进行计算,
但是目前MS@GWEV仅支持对\textsc{Gaussian} 16程序的调用。

\subsection{编译MS@GWEV}

进入\verb|MultiState/src|,用\textsl{make}命令调用\textsf{gfortran}进行编译:
\begin{lstlisting}[language=bash,numbers=none,backgroundcolor=\color{black},basicstyle=\tt\small\color{white},deletekeywords={cd}]
$ cd MultiState/src/
$ make
\end{lstlisting}

如果用\textsf{ifort}+\textsf{MKL}库编译,需要用\verb|Makefile-intel|覆盖\verb|Makefile|,并修改其中的MKL路径\textsf{MKLROOT}。
此时不需要\verb|blas.f|和\verb|lapack.f|这两个文件。

编译完成后,程序主目录\verb|MultiState|的内容如下:
\begin{lstlisting}[language=bash,numbers=none,backgroundcolor=\color{black},basicstyle=\tt\small\color{white},
alsoletter={.-},deletekeywords={cd},
classoffset=0,morekeywords={run-2state.sh, run-3state.sh, multistate.exe},keywordstyle=\color{green},
classoffset=1,morekeywords={doc, src, tests},keywordstyle=\color{blue!50!white},
classoffset=0]
$ cd ..
$ ls -l
total 464
drwxrwxrwx 1 zouwl zouwl   4096 Nov  1 00:51 doc/
-rwxrwxrwx 1 zouwl zouwl 455143 Nov  1 00:51 multistate.exe
-rwxrwxrwx 1 zouwl zouwl    930 Oct 31 11:05 run-2state.sh
-rwxrwxrwx 1 zouwl zouwl   1130 Oct 31 11:06 run-3state.sh
drwxrwxrwx 1 zouwl zouwl   4096 Oct 31 17:16 src/
-rwxrwxrwx 1 zouwl zouwl   3266 Oct 31 18:19 templet-fes
-rwxrwxrwx 1 zouwl zouwl   2241 Apr  3  2022 templet-o2
drwxrwxrwx 1 zouwl zouwl   4096 Oct 31 18:30 tests/
\end{lstlisting}
其中\verb|run-2state.sh|、\verb|run-3state.sh|、\verb|multistate.exe|需要具有可执行权限。

\subsection{运行MS@GWEV和\textsc{Gaussian} 16的脚本}

以两态计算为例(见文件\verb|MultiState/run-2state.sh|):
\begin{lstlisting}[language=bash,morekeywords={module, mkdir}]
#!/bin/bash

# Gaussian 16
module load chemsoft/g16c-avx
export GAUSS_SCRDIR=/tmp/zouwl/GaussianScr

mkdir -p $GAUSS_SCRDIR

export gaussian_ein=$2
export gaussian_eou=$3

# MS@GWEV
export CurrDir=`pwd`
export multistate_dir=$CurrDir/MultiState
export ms_scrdir=$CurrDir/JOB001
mkdir -p $ms_scrdir
export tem_mstate=$multistate_dir/templet-o2
export inp_state1=$ms_scrdir/state1.gjf
export inp_state2=$ms_scrdir/state2.gjf
export fch_state1=$ms_scrdir/state1.fch
export fch_state2=$ms_scrdir/state2.fch

# generate input files for states 1 and 2
$multistate_dir/multistate.exe -gen -gin $gaussian_ein -ctp $tem_mstate \
  -in1 $inp_state1 -in2 $inp_state2

rm -f $fch_state1 $fch_state2

# state 1 calculation
g16 -fchk=$fch_state1 $inp_state1

# state 2 calculation
g16 -fchk=$fch_state2 $inp_state2

# write Gaussian's *.EOu file
$multistate_dir/multistate.exe -mix -chi 400 -gin $gaussian_ein -gou $gaussian_eou \
  -fc1 $fch_state1 -fc2 $fch_state2
\end{lstlisting}
脚本分为5部分:
\begin{itemize}
\item 23行以前,是加载\textsc{Gaussian}、MS@GWEV有关的环境变量。
\item 23-25行,MS@GWEV根据*.EIn(主计算产生)和模板文件(内容见下一节)的信息,产生态1、态2的子计算输入文件。
\item 27行,删除旧的fchk文件。如果在结构优化和数值频率计算中出现SCF不收敛,fchk文件将不会更新,此时没被删除的旧fchk文件会对计算造成干扰。
\item 29-33行,运行态1、态2的\textsc{Gaussian}计算,产生fchk文件(通过\textsc{Gaussian} 16新加的命令行选项\verb|-fchk|)。未来可能支持其它量子化学程序。
\item 35-37行,MS@GWEV根据*.EIn和两个fchk文件中的信息,产生*.EOu文件返回给\textsc{Gaussian}主计算。其中可以通过\verb|-chi|选项设置非默认的旋轨耦合参数$\chi$(单位:cm$^{-1}$)。
\end{itemize}

对于两个以上自旋态的计算,需要用\verb|-nst|选项指定态的个数,并用\verb|-in3|、\verb|-fc3|等选项为态3指定相应的文件。
见文件\verb|MultiState/run-3state.sh|。

\subsection{两个自旋态的\textsc{Gaussian}计算模板}

\ce{O2}分子的模板见文件MultiState/templet-o2。
\begin{lstlisting}[alsoletter={$*},
morekeywords={$sp1,$grad1,$freq1,$sp2,$grad2,$freq2,*before_geom,*after_geom,*end_of_input}]
! templet of G16 input: singlet and triplet state of O2 by DFT/TDDFT

! ----------------------------------------------------------
! Single point energy of the first state
!
! This step is used to generate a checkpoint file, so you
! must do SP calculation first.
! ----------------------------------------------------------
$sp1
  *before_geom
%mem=8GB
%nprocshared=4
%chk=o2-s1
#p b3lyp/3-21g

Title: singlet state of O2

0 1
  *end_of_input

  *after_geom

  *end_of_input

! ----------------------------------------------------------
! Single point energy of the second state
!
! This step is used to generate a checkpoint file, so you
! must do SP calculation first.
! ----------------------------------------------------------
$sp2
  *before_geom
%mem=8GB
%nprocshared=4
%chk=o2-s2
#p b3lyp/3-21g td(triplets,root=1)

Title: triplet state of O2

0 1
  *end_of_input

  *after_geom

  *end_of_input

! ----------------------------------------------------------
! Gradients of the first state
! ----------------------------------------------------------
$grad1
  *before_geom
%mem=8GB
%nprocshared=4
%chk=o2-s1
#p b3lyp/3-21g guess=read force

Title: singlet state of O2

0 1
  *end_of_input

  *after_geom

  *end_of_input

! ----------------------------------------------------------
! Gradients of the second state
! ----------------------------------------------------------
$grad2
  *before_geom
%mem=8GB
%nprocshared=4
%chk=o2-s2
#p b3lyp/3-21g guess=read td(triplets,root=1) force

Title: triplet state of O2

0 1
  *end_of_input

  *after_geom

  *end_of_input

! ----------------------------------------------------------
! Hessians of the first state
! ----------------------------------------------------------
$freq1
  *before_geom
%mem=8GB
%nprocshared=4
%chk=o2-s1
#p b3lyp/3-21g guess=read freq

Title: singlet state of O2

0 1
  *end_of_input

  *after_geom

  *end_of_input

! ----------------------------------------------------------
! Hessians of the second state
! ----------------------------------------------------------
$freq2
  *before_geom
%mem=8GB
%nprocshared=4
%chk=o2-s2
#p b3lyp/3-21g guess=read td(triplets,root=1) freq

Title: triplet state of O2

0 1
  *end_of_input

  *after_geom

  *end_of_input

\end{lstlisting}
注释:
\begin{itemize}
\item 以``!''打头的行,是注释行,会被程序略过。
\item 模板分为6部分,分别以\verb|$sp1|、\verb|$sp2|、\verb|$grad1|、\verb|$grad2|、\verb|$freq1|、\verb|$freq2|作为开头,表示第1或第2个态的单点能、梯度、频率计算部分。不区分大小写。 \\
    类似地,可以定义第3、第4等态的单点能、梯度、频率计算部分(见\verb|MultiState/templet-fes|)。
\item 在模板的每一部分中,又包含两块输入,分别用\verb|*before_geom| $\cdots$ \verb|*end_of_input|和\verb|*after_geom| $\cdots$ \verb|*end_of_input|包围。前者提供态1或态2位于分子结构之前的单点/梯度/频率输入内容。后者提供分子结构结束空行之后的输入内容(如基组,赝势),不是必须的。这两块输入的格式与\textsc{Gaussian}的标准输入格式相同。
\item 在结构优化和频率计算中,为了避免电子组态发生变化导致不收敛或收敛到其它电子态上,建议加上\verb|guess=read|,表示从\verb|%chk|指定的chk文件中读取上一步的波函数作为初猜。 \\
    \textbf{注意:}这里的chk文件和上一节运行脚本中指定的fchk文件包含相同的信息,但是格式不同,用途也不同,不要混淆。
\item 本例中,态1是\ce{O2}的闭壳层激发态,通过限制性DFT计算产生,态2是\ce{O2}的开壳层三重基态,在闭壳层激发态DFT的基础上由自旋翻转TDDFT计算产生,当然也可以直接做三重态的UDFT。自旋混合后,得到近似的旋轨耦合基态。
\end{itemize}

\subsection{在\textsc{Gaussian}中调用运行脚本}

\ce{O2}分子的测试输入见文件\verb|MultiState/tests/test-2state.inp|。
\begin{lstlisting}[alsoletter={\%\#},morekeywords={\%nprocshared,\%mem,\#p,gb,external}]
%nprocshared=1
%mem=1gb
#p external='MultiState/run-2state.sh'
!  opt(nomicro)
!  freq

Two-State calculation (DO SP FIRST.)

0  1
O
O 1 r1

r1 1.2
\end{lstlisting}
注释:
\begin{itemize}
\item 主计算的计算量很小(500+原子体系的频率计算除外),只要单核1GB内存即可。
\item 用\textsf{External}调用计算脚本。
\item 计算采用的分子净电荷、自旋多重度已经在计算模板中设定,这里可以用任何\textsc{Gaussian}程序允许的设置。
\item \textsf{External}默认的结构优化流程是专门为分子力学计算设计的,不适合做量子力学计算,需要给\verb|opt|关键词加上\verb|nomicro|选项。
\item 由于\verb|opt|和\verb|freq|的模板中用到了\verb|guess=read|,需要先运行一次单点能计算,使产生的chk文件中包含所需的初始轨道。此时可以检查两个自旋态子计算的输出文件,看电子组态、总能量等是否符合要求;对于具有一定对称性的过渡金属体系,\textsc{Gaussian}未必能得到给定自旋的最低态,因此检查波函稳定性也非常重要。然后去掉\verb|opt|(必要的话还有\verb|freq|)之前的注释符号,重新运行。
\item 可以在主计算中用\verb|%chk|指定chk文件(不要与子计算中的chk文件重名),用于保存混合态的一些重要信息,如优化自旋混合过渡态和反应路径所需的初始Hessian。
\end{itemize}

\subsection{MS@GWEV程序的参数}

以下两个参数控制计算模式,如果不指定,默认为\verb|-gen|。 \\
\verb|-gen|:产生自旋态的\textsc{Gaussian}输入文件 \\
\verb|-mix|:计算自旋混合基态的能量,梯度,和Hessian,并返回给\textsc{Gaussian}主计算

\bigskip
其它参数有:  \\
\verb|-nst| \textit{N}:进行混合的自旋态个数,允许值2至9;如果不提供,默认是2 \\
\verb|-gin| \textit{path/file\_name}:指定\textsc{Gaussian}的*\verb|.EIn|文件名及其路径 \\
\verb|-gou| \textit{path/file\_name}:指定\textsc{Gaussian}的*\verb|.EOu|文件名及其路径 \\
\verb|-ctp| \textit{path/file\_name}:指定模板文件名及其路径 \\
\verb|-in1| \textit{path/file\_name}:指定自旋态1的输入文件名及其路径;类似有\verb|-in2|,\verb|-in3|,等 \\
\verb|-fc1| \textit{path/file\_name}:指定自旋态1的\verb|fchk|文件名及其路径;类似有\verb|-fc2|,\verb|-fc3|,等 \\
\verb|-chi| \textit{$\chi$}:经验的旋轨耦合常数(单位cm$^{-1}$),正负无关;如果不指定,默认是400 \\
\verb|-chs| \textit{$\chi_{1,2}$ $\chi_{1,3}$ $\ldots$}:每对自旋态之间的旋轨耦合常数(单位cm$^{-1}$;注意正负号),共\textit{N}(\textit{N}-1)/2个值 \\
\verb|-sh1| \textit{$\delta_1$}:指定自旋态1的能级移动(单位cm$^{-1}$);类似有\verb|-sh2| \textit{$\delta_2$},\verb|-sh3| \textit{$\delta_3$},等;如果不指定,默认是0

\bigskip
以上所有参数都不区分大小写。

\bigskip
在\verb|-gen|模式下,涉及的参数有:\\
\verb|-nst|,\verb|-gin|,\verb|-ctp|,\verb|-in1|,$\ldots$,\verb|-in|\textit{N}。

\bigskip
在\verb|-mix|模式下,涉及的参数有:\\
\verb|-nst|,\verb|-gin|,\verb|-gou|,\verb|-chi|(或\verb|-chs|),\verb|-fc1|,$\ldots$,\verb|-fc|\textit{N},\verb|-sh1|,$\ldots$,\verb|-sh|\textit{N}。


\newpage

\section{常见问题}

\begin{itemize}
%\item 存在两个以上的自旋轨道态 \\
%  如果反应路径涉及两个以上的自旋轨道态,可以分段、分组做两态计算,然后拼接在一起,取能量最低路径。示例见Comput. Theor. Chem. \textbf{1211}, 113682 (2022)。\\
%  \textbf{本程序不适合处理两个以上自旋轨道态在同一位置交叉的情况!}
\item 主计算报错为某原子的基组找不到

  实际计算采用的基组在计算模板中定义,而在主计算输入中一般不需要指定基组。此时\textsc{Gaussian}假设主计算采用STO-3G基组,适用于H-Xe原子,
  若遇到STO-3G不支持的重原子便会报错。解决方法是在主计算输入中指定支持较多元素类型的\textsf{SDD}、\textsf{UGBS}基组,也可通过\textsf{Gen}定义额外的基组信息(不必用\textsf{GenECP}),其中的基组内容可随意写,能通过检测即可。\\
  \textbf{注意:} 在调用\textsf{External}的情况下,\textsc{Gaussian}无法在多步任务中传递基组。因此\verb|Opt|和\verb|Freq|必须分开计算,或者用\verb|--Link1--|在主计算的单个输入文件中定义多步任务,其中每步计算都要定义基组。

\item 如何估算旋轨耦合参数?

见文献\cite{ref3}的ESI。此外,有些量子化学软件(如\textsc{Molpro})可以分别打印一阶和二阶旋轨耦合常数。
对于重原子成键饱和的反应,结果对旋轨耦合参数的设置不敏感。

\end{itemize}

\newpage

\appendix

\section{附录:绘制两态旋轨耦合系数、能量差与组合因子的\textsc{Matlab}代码}

\begin{lstlisting}[language=matlab]
% chi = 2400 cm-1
x=2400/219474.63137;
% dE = abs(Vm - Vn) = 0-2 eV
de=0:0.1:2;
%
% starting calculations
deh=de/27.2114;
b=sqrt(deh.*deh+4*x*x);
a=deh./b;
c1=(1-a)*0.5;c2=(1+a)*0.5;c3=(a.*a-1)*0.5./b;
plot(de,c1,'r-o',de,c2,'b-o')
%plot(de,c3,'b-o')
\end{lstlisting}

以上\textsc{Matlab}代码若用于\textsc{Scilab},需删除注释行,或把首列的“\%”替换为“\#”。

\end{document}
