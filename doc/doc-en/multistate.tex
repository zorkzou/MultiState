\documentclass[UTF8]{article}
\usepackage{graphicx}
\usepackage{float}
\usepackage{mhchem}
\usepackage{amsmath}
\usepackage{enumitem}
\usepackage{geometry}
\geometry{a4paper,centering,scale=0.8}
\usepackage[format=hang,font=small,textfont=it]{caption}
\usepackage[nottoc]{tocbibind}

\usepackage{listings,xcolor}
\lstset{
%        extendedchars=false,          % non-English characters
        lineskip=3pt,
        basicstyle=\tt\small\color{black},
        commentstyle=\tt\color{red!60!black},
        morecomment=[l]\!,
        keywordstyle=\tt\color{green!70!black},
        stringstyle=\tt\color{green},
        showspaces=false,             % underline spaces in the codes
        showstringspaces=false,       % underline spaces only in a string
        showtabs=false,               % underline tabs in the codes
        numberstyle=\tiny\color{black},
        numbersep=14pt,               % how far the line-numbers are from the code
        texcl=true,                   % comments in LaTeX if true
        emph={                        % keywords to be highlighted
%               subroutine, return,
%               end
        },
        emphstyle=\sf\bfseries\color{red!50!black}\fcolorbox{orange!40!white}{.},
        numbers=left,
        rulecolor=\color{blue},
        frame=single
%        tabsize=2,                    % set default tab-size to 2 spaces
%        frame=shadowbox, rulesepcolor=\color{blue}
}

\newcommand\degree{^\circ}

\title{MS@GWEV: An Interface Program of Approximate Spin-Orbit Coupling for Spin-Forbidden Reactions}

\date{\today}

\begin{document}

\maketitle

\vspace{10mm}
\tableofcontents

\newpage

The MS@GWEV program is a module of the in-house software \textsc{GWEV Suite} developed in our group.
It was written in Fortran 90 and interfaced to \textsc{Gaussian} 16 (through the keyword \textsf{External}) to calculate spin-forbidden reactions involving multiple spin states.

\section{Installation}

If the \textsf{gfortran} compiler has been installed, MS@GWEV may be built by the \textsl{make} command in the \verb|MultiState/src| directory,
\begin{lstlisting}[language=bash,numbers=none,backgroundcolor=\color{black},basicstyle=\tt\small\color{white},deletekeywords={cd}]
$ cd MultiState/src/
$ make
\end{lstlisting}

If the \textsf{ifort} compiler with the \textsf{MKL} library is preferred, the user may replace \verb|Makefile| by \verb|Makefile-intel| and modify \textsf{MKLROOT} therein.

After the installation, the user will see the following contents in the \verb|MultiState| main directory:
\begin{lstlisting}[language=bash,numbers=none,backgroundcolor=\color{black},basicstyle=\tt\small\color{white},
alsoletter={.-},deletekeywords={cd},
classoffset=0,morekeywords={run-2state.sh, run-3state.sh, multistate.exe},keywordstyle=\color{green},
classoffset=1,morekeywords={doc, src, tests},keywordstyle=\color{blue!50!white},
classoffset=0]
$ cd ..
$ ls -l
total 464
drwxrwxrwx 1 zouwl zouwl   4096 Nov  1 00:51 doc/
-rwxrwxrwx 1 zouwl zouwl 455143 Nov  1 00:51 multistate.exe
-rwxrwxrwx 1 zouwl zouwl    930 Oct 31 11:05 run-2state.sh
-rwxrwxrwx 1 zouwl zouwl   1130 Oct 31 11:06 run-3state.sh
drwxrwxrwx 1 zouwl zouwl   4096 Oct 31 17:16 src/
-rwxrwxrwx 1 zouwl zouwl   3266 Oct 31 18:19 templet-fes
-rwxrwxrwx 1 zouwl zouwl   2241 Apr  3  2022 templet-o2
drwxrwxrwx 1 zouwl zouwl   4096 Oct 31 18:30 tests/
\end{lstlisting}
where \verb|run-2state.sh|, \verb|run-3state.sh|, and \verb|multistate.exe| have executable permissions.

\section{Script to run MS@GWEV}

The following sample script (\textit{i.e.} the script file \verb|MultiState/run-2state.sh|) calculates the mixed-spin ground state from two spin states.
The two spin states are computed by \textsc{Gaussian} 16, whereas the other quantum chemistry programs may be supported in the future.
For the calculation of three spin states, please refer to the script \verb|MultiState/run-3state.sh|.

\begin{lstlisting}[language=bash,morekeywords={module, mkdir}]
#!/bin/bash

# Gaussian 16
module load chemsoft/g16c-avx
export GAUSS_SCRDIR=/tmp/zouwl/GaussianScr

mkdir -p $GAUSS_SCRDIR

export gaussian_ein=$2
export gaussian_eou=$3

# MS@GWEV
export CurrDir=`pwd`
export multistate_dir=$CurrDir/MultiState
export ms_scrdir=$CurrDir/JOB001
mkdir -p $ms_scrdir
export tem_mstate=$multistate_dir/templet-o2
export inp_state1=$ms_scrdir/state1.gjf
export inp_state2=$ms_scrdir/state2.gjf
export fch_state1=$ms_scrdir/state1.fch
export fch_state2=$ms_scrdir/state2.fch

# generate input files for states 1 and 2
$multistate_dir/multistate.exe -gen -gin $gaussian_ein -ctp $tem_mstate \
  -in1 $inp_state1 -in2 $inp_state2

rm -f $fch_state1 $fch_state2

# state 1 calculation
g16 -fchk=$fch_state1 $inp_state1

# state 2 calculation
g16 -fchk=$fch_state2 $inp_state2

# write Gaussian's *.EOu file
$multistate_dir/multistate.exe -mix -chi 400 -gin $gaussian_ein -gou $gaussian_eou \
  -fc1 $fch_state1 -fc2 $fch_state2
\end{lstlisting}

The script contains five parts:
\begin{itemize}
\item Before the line 23, the environment variables of \textsc{Gaussian} and MS@GWEV are loaded.
\item Lines 23 to 25, MS@GWEV generates input files of the two spin states according to the *.EIn (from the master calculation) and template (see the next section) files.
\item Line 27, delete the old fchk files. If SCF does not converge during geometry optimization or numerical frequency calculations, a new fchk file will not be saved. In this case the old fchk file if not deleted may lead to strange results.
\item Lines 29 to 33, the two spin states are calculated by \textsc{Gaussian}, and two fchk files will be saved through the new command line option \verb|-fchk| of \textsc{Gaussian} 16.
\item Lines 35 to 37, MS@GWEV calculates energy, gradients, and possible Hessians of the mixed-spin ground state by using the data from *.EIn and two fchk files, and returns them to the \textsc{Gaussian} master calculation through the data file *.EOu.
    The non-default empirical SOC constant $\chi$ may be set by the option \verb|-chi|.
\end{itemize}

\section{\textsc{Gaussian} 16 Template with Two Spin States}

This is a \textsc{Gaussian} 16 template to do two-state calculation of \ce{O2} (see MultiState/templet-o2).
\begin{lstlisting}[alsoletter={$*},
morekeywords={$sp1,$grad1,$freq1,$sp2,$grad2,$freq2,*before_geom,*after_geom,*end_of_input}]
! templet of G16 input: singlet and triplet state of O2 by DFT/TDDFT

! ----------------------------------------------------------
! Single point energy of the first state
!
! This step is used to generate a checkpoint file, so you
! must do SP calculation first.
! ----------------------------------------------------------
$sp1
  *before_geom
%mem=8GB
%nprocshared=4
%chk=o2-s1
#p b3lyp/3-21g

Title: singlet state of O2

0 1
  *end_of_input

  *after_geom

  *end_of_input

! ----------------------------------------------------------
! Single point energy of the second state
!
! This step is used to generate a checkpoint file, so you
! must do SP calculation first.
! ----------------------------------------------------------
$sp2
  *before_geom
%mem=8GB
%nprocshared=4
%chk=o2-s2
#p b3lyp/3-21g td(triplets,root=1)

Title: triplet state of O2

0 1
  *end_of_input

  *after_geom

  *end_of_input

! ----------------------------------------------------------
! Gradients of the first state
! ----------------------------------------------------------
$grad1
  *before_geom
%mem=8GB
%nprocshared=4
%chk=o2-s1
#p b3lyp/3-21g guess=read force

Title: singlet state of O2

0 1
  *end_of_input

  *after_geom

  *end_of_input

! ----------------------------------------------------------
! Gradients of the second state
! ----------------------------------------------------------
$grad2
  *before_geom
%mem=8GB
%nprocshared=4
%chk=o2-s2
#p b3lyp/3-21g guess=read td(triplets,root=1) force

Title: triplet state of O2

0 1
  *end_of_input

  *after_geom

  *end_of_input

! ----------------------------------------------------------
! Hessians of the first state
! ----------------------------------------------------------
$freq1
  *before_geom
%mem=8GB
%nprocshared=4
%chk=o2-s1
#p b3lyp/3-21g guess=read freq

Title: singlet state of O2

0 1
  *end_of_input

  *after_geom

  *end_of_input

! ----------------------------------------------------------
! Hessians of the second state
! ----------------------------------------------------------
$freq2
  *before_geom
%mem=8GB
%nprocshared=4
%chk=o2-s2
#p b3lyp/3-21g guess=read td(triplets,root=1) freq

Title: triplet state of O2

0 1
  *end_of_input

  *after_geom

  *end_of_input

\end{lstlisting}

A few comments on the template:
\begin{itemize}
\item A line beginning with an exclamation mark (!) is a comment line, which is ignored by MS@GWEV.
\item The template has six parts beginning with the keywords \verb|$sp1|, \verb|$sp2|, \verb|$grad1|, \verb|$grad2|, \verb|$freq1|, and \verb|$freq2| (case insensitive),
    indicating the sing-point energy, gradient, and vibrational frequency calculation sections of the first and the second spin states, respectively.
    Similarly, the keywords like \verb|$sp3| and \verb|$grad4| may be defined for the other states (see \verb|MultiState/templet-fes|).
\item In each part of the template, there are two input fields, enclosed in the opening-closing tags \verb|*before_geom| $\cdots$ \verb|*end_of_input| and \verb|*after_geom| $\cdots$ \verb|*end_of_input|, respectively.
    The text content in the former is the \textsc{Gaussian} input stream for sing-point energy, gradient, or vibrational frequency calculation of a spin state before the molecular geometry,
    and in the latter is the optional \textsc{Gaussian} input stream after the molecular geometry, such as basis sets and pseudo-potentials.
\item In the gradient and vibrational frequency calculations, the keyword \verb|guess=read| is suggested to use, which reads wavefunction of previous step from the chk file (defined via the \verb|%chk| command; \textbf{Don't confuse it with the fchk file in the script!}) as initial guess.
\item In this example, the spin state 1 is a closed-shell excited state of \ce{O2} computed by restricted DFT, whereas the spin state 2 is the open-shell triplet ground state of \ce{O2}, computed by spin-flip TDDFT (of course unrestricted DFT with the spin multiplicity being 3 is also available).
\end{itemize}

\section{Calling the script in \textsc{Gaussian} master calculation}

This is the test input of \ce{O2} (see \verb|MultiState/tests/test-2state.inp|).
\begin{lstlisting}[alsoletter={\%\#},morekeywords={\%nprocshared,\%mem,\#p,gb,external}]
%nprocshared=1
%mem=1gb
#p external='MultiState/run-2state.sh'
!  opt(nomicro)
!  freq

Two-State calculation (DO SP FIRST.)

0  1
O
O 1 r1

r1 1.2
\end{lstlisting}

A few comments on the input:
\begin{itemize}
\item The master calculation does not consume too much memory and CPU resources (except the frequency calculation with more than 500 atoms),
so one processor with 1 GB of memory is usually enough.
\item The keyword \textsf{External} is used to call the script.
\item Since the charge and spin multiplicity have been set in the template file, here they can be any accepted values.
\item The default optimization procedure of \textsf{External} is designed for MM calculations but is not suitable for QM calculations, so the option \verb|nomicro| is needed for \verb|opt|.
\item Since \verb|guess=read| is specified in the template file for \verb|opt| and \verb|freq| calculations, a single-point calculation
    has to be performed first to save wavefunctions into the checkpoint files.
    After the single-point calculation, the user may check electronic configurations, total energies, and populations in the output files of the two spin states; for transition-metal systems with some symmetry, it is highly suggested to check the stabilities of wavefunction since \textsc{Gaussian}
    may not converge to the lowest solution correctly.
    Then the user may remove the exclamation mark symbol before \verb|opt| and \verb|freq| and redo the calculation.
\item A checkpoint file may be specified by \verb|%chk| in the master calculation to save some important data of the mixed-spin ground state,
    \textit{e.g.} initial Hessians for subsequent TS and IRC optimization.
\end{itemize}

\section{Options of MS@GWEV}

\subsection{Options to Control Operating Mode}

The following options are case insensitive.

\begin{itemize}[leftmargin= 0 pt]
\item \verb|-gen|

This mode generates \textsc{Gaussian} input files of spin states, which is the default.
The options involved are \verb|-nst|, \verb|-gin|, \verb|-ctp|, \verb|-in1|, $\ldots$, \verb|-in|\textit{N}.

\item \verb|-mix|

This mode computes energy, gradient array, and Hessian matrix of the mixed-spin ground state, and returns them to the \textsc{Gaussian} master calculation.
The options involved are \verb|-nst|, \verb|-gin|, \verb|-gou|, \verb|-chi| (or \verb|-chs|), \verb|-fc1|, $\ldots$, \verb|-fc|\textit{N}, \verb|-sh1|, $\ldots$, \verb|-sh|\textit{N}.

\end{itemize}

\subsection{Other Options}

\begin{itemize}[leftmargin= 0 pt]
\item \verb|-nst| \textit{N}

This option specifies the number of spin electronic states to be mixed, which is between 2 and 9.
 The default is 2 if \verb|-nst| is not specified.

\item \verb|-gin| \textit{path/file\_name}

This option specifies the path and name of \textsc{Gaussian}'s *\verb|.EIn| file.

\item \verb|-gou| \textit{path/file\_name}

This option specifies the path and name of \textsc{Gaussian}'s *\verb|.EOu| file.

\item \verb|-ctp| \textit{path/file\_name}

This option specifies the path and name of the template file.

\item \verb|-in1| \textit{path/file\_name} \\
 \verb|-in2| \textit{path/file\_name} \\
 $\ldots$ \\
 \verb|-in|\textit{N} \textit{path/file\_name}

These options specify the paths and names of the input files for spin states 1, 2, $\ldots$, $N$.

\item \verb|-fc1| \textit{path/file\_name} \\
 \verb|-fc2| \textit{path/file\_name} \\
 $\ldots$ \\
 \verb|-fc|\textit{N} \textit{path/file\_name}

These options specify the paths and names of the \verb|fchk| files for spin states 1, 2, $\ldots$, $N$.

\item \verb|-chi| \textit{$\chi$}

This option specifies the empirical SO constant (in cm$^{-1}$). The default is 400 cm$^{-1}$ if \verb|-chi| is not specified.
Taking three spin states as an example, the constructed model SO Hamiltonian is
\begin{align}\label{hso0-eq1}
\mathbf{H} = \left[\begin{array}{ccc}
    V_1 & -|\chi| & -|\chi| \\
    -|\chi| & V_2 & -|\chi| \\
    -|\chi| & -|\chi| & V_3
\end{array}\right]
\end{align}

\item \verb|-chs| \textit{$\chi_{1,2}$ $\chi_{1,3}$ $\ldots$ $\chi_{2,3}$ $\ldots$}

This option must appear after \verb|-nst| \textit{N}, which specifies the empirical SO constant (in cm$^{-1}$) of each pair of spin states, and totally \textit{N}(\textit{N}-1)/2
values should be provided.
Taking three spin states as an example, the constructed model SO Hamiltonian is
\begin{align}\label{hso0-eq2}
\mathbf{H} = \left[\begin{array}{ccc}
    V_1 & \chi_{1,2} & \chi_{1,3} \\
    \chi_{1,2} & V_2 & \chi_{2,3} \\
    \chi_{1,3} & \chi_{2,3} & V_3
\end{array}\right]
\end{align}

\item \verb|-sh1| \textit{$\delta_1$} \\
 \verb|-sh2| \textit{$\delta_2$} \\
 $\ldots$ \\
 \verb|-sh|\textit{N} \textit{$\delta_N$}

These options specify energy shifts (in cm$^{-1}$) of spin states 1, 2, $\ldots$, $N$.
The default values are all 0 cm$^{-1}$ if they are not specified.
Taking three spin states as an example, the modified diagonal elements in the model SO Hamiltonian are
\begin{align}\label{hso0-eq2}
\mathbf{H} = \left[\begin{array}{ccc}
    V_1 + \delta_1 & {} & {} \\
    {} & V_2 + \delta_2 & {} \\
    {} & {} & V_3 + \delta_3
\end{array}\right] \> .
\end{align}

\end{itemize}


\section{Frequently Asked Questions}

\begin{itemize}
\item \textbf{The \textsc{Gaussian} master calculation complains that the basis set is not available for an atom.}

The actual basis set used in the calculation is defined in the template file, whereas in the master \textsc{Gaussian} input file
the basis set is usually not needed to specify. In this case \textsc{Gaussian} assumes that the \textsf{STO-3G} basis set is used,
which supports the atoms H-Xe. Thus if there are heavier atoms than Xe, \textsc{Gaussian} reports an error.

To solve this problem, one may specify either the \textsf{SDD} or \textsf{UGBS} basis set, which support more atoms,
or custom basis functions through the \textsf{Gen} keyword (\textsf{GenECP} is not necessary).

Please note that, in the case of \textsf{External} calculation mode, the non-default basis set in the master input
cannot be passed on to the subsequent steps in a multiple-step \textsc{Gaussian} job.
Thus \textsf{Opt} and \textsf{Freq} have to be computed separately; they may also be combined in a single master input
through \verb|--Link1--|.

\item \textbf{How to estimate the SOC constant?}

Please refer to the ESI of \textbf{Phys. Chem. Chem. Phys.} 20, 4129, 2018.
Some quantum chemistry programs (\textit{e.g.} \textsc{Molpro}) can print the first- and the second-order SOC constants separately.
For the reaction where the heavy atom has saturated chemical bonds, the results are not sensitive to the SOC constant.

\end{itemize}

\end{document}
